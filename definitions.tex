\section{Entropy and Trace Predictability}

\subsection{Entropy and Conditional Entropy}

% TODO: something abt the entropy of a trace as a stochastic process

\subsection{$n$-gram Entropy}

\textit{Temporal prefetching} remembers high-frequency sequences of
consecuitive accesses and prefetches subsequent items when prefixes of
sequences occur. The \textit{$n$-gram entropy} of a trace provides a natural
quantification of the temporal prefetchability of the trace.

% TODO: cite about temporal prefetching?

\begin{definition}[$n$-gram] Let $n$ be a positive integer. Then an $n$-gram is
an $n$-length sequence of (not necessarily unique) pages. We use $w_t^n$ to
refer to the the set of all $n$-grams which appear in the trace $t$.
\end{definition}

Temporal prefetching essentially works by memorization of common $n$-grams, so
they are a natural starting point for examining the temporal prefetcability of
a trace. In particular, we are interested in the entropy associated with the
subsequent page requested after any particular $n$-gram; a lower entropy
implies greater predictability for that specific $n$-gram.

\begin{definition} Let $b$ be an $n$-gram and $t$ a trace. Then $H(t \mid b)$
  refers to the entropy of $t$ conditioned on seeing $b$, i.e. the entropy of
  the subset of requests which occur immediately after requests to the $n$
  pages in $b$. Further, we use $P_t(b)$ for the experimentally observed
probability of seeing $b$ in any particular $n$-length window of $t$.
\end{definition}



% TODO: representing conditions that could occur a smaller numebr of times than
% |t|, i.e. windows. (Maybe just define stride entropy?)

% TODO: is there a probability notation for this?

\begin{definition}[$n$-gram entropy] For some positive integer n, the
  \textit{$n$-gram entropy} of a trace $t$ is

\begin{equation}
  H_{w_n}(t) = \sum_{b \in w^n_t} P_t(b) \cdot H(t \mid b).
\end{equation}

\end{definition}

% TODO: do this for strides, not just last-n. One idea is to define $s_t$ or
% similar as the sequence of strides (consecutive differences) in $t$, and then
% define terms similarly to above.
